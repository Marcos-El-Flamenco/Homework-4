%Colle du Mercredi 28 a Janson, edp
\documentclass{scrartcl}

\usepackage{cmap}
\usepackage{lmodern}
\usepackage[T1]{fontenc}
\usepackage[french]{babel}
\usepackage{ marvosym }
\usepackage{amsmath}
\usepackage{amsfonts}
\usepackage{amssymb}

\title{Homework 4 Exercise 1}
\author{Oscar Garnier}
\date{\today}


\begin{document}
\newcommand{\E}[1]{\section*{Exo #1}}
\newcommand{\CR}[2]{\section*{#1 // note : #2}}
\newcommand{\Q}[1]{\section*{Exercise #1}}
\newcommand{\SQ}[1]{\subsection*{Question #1}}
\maketitle

For simplicity, we will assimilate the notations \( C, E(C) \) and \( V(C) \) when no ambiguity comes of it.

\SQ{1}

\( C \) contains an even number of edges, since half of them are supposed to be in \( M \).\\
Let \( u \in V \).
Since \( M \) is a perfect matching in \( G \), \( \exists ! e \in E\) s.t. \( e \in M \) and \( u \in e \).
\begin{enumerate}
\item If \( e \not \in C \) then \( e \in M \Delta E(C) \) and \( u \) is still uniquly matched in \( M' \) in \( M' \) 
\item If \( e \in C \), then \( \exists ! e' \in C \) s.t. \( u \in e' \), so \( e \) and \( e' \) are adjacent, which means \( e' \not \in M \). Hence \( e \not \in M'\) and \( e' \in M' \). So \( u \) is still uniquly matched in \( M' \).
\end{enumerate}

\SQ{2}
Let us denote \( c(M) = \sum_{e \in M}c(e) \). Notice that, given \( M \) a matching and \( C \) an alternating cycle, \( c(M' = M \Delta C) = c(M) - c(C \cap M) + c(C / M) \).
Let \( M \) be a minimum cost perfect matching, and \( C \) be an M-alternating cycle in \( G \). Since \( M \) is of minimal weight, \( c(M' = M \Delta C) \geq c(M) \). Hence \[
c(M) - c(C \cap M) + c(C / M) \geq c(M) \Rightarrow c(C \cap M) \leq c( C / M ) \]
So \( C \) is not negative. \( G \) has no negative M-alternating cycles. \\


Let \( M \) be a perfect matching, and \( M' \) be some other perfect matching. Let us show that \( C = M \Delta M' \) is a set of M-alternating cycles. \\
Let \( u \in E \). \( \exists ! a \in M s.t. u \in a \) and \( \exists ! b \in M s.t. u \in b \). If \( u \in V(C) \) (where \( V(C)\) is defined as the set of endpoints of edges in \( C \) ), this means \( a \neq b \). So \( u \) is the endpoint of exactly one edge from \( M \) and one other edge from \( M' \). 
By iterating this proprety, \( u \) is part of an M-alternating cycle. 
There are then as many of these as there are connexe components in \( C \).

Now suppose \( G \) has no negative weight M-alternating cycles. Since \( C = M \Delta M', M' = M \Delta C \), hence \( c(M') = c(M) + c(C / M) - c(C \cap M) \). And since \( C \) is composed of M-alternating cycles, each non-negative, \( \forall C_i \) connected component of \( C \), \( c(C_i / M) - c(C_i \cap M) \geq 0\), by summing these inequalities we get \( c(C / M) - c(C \cap M) \geq 0 \), hence \( c(M') \geq c(M) \). \( M \) is in fact a minimal cost perfect matching.

\SQ{3}

If \( M \) is not a minimum cost perfect matching, according to the previous question, there exists a negative M-alternating cycle C in \( G \). But \[ c(C \cap M) = \sum{(x,y) \in C \cap M}c((x,y)) = \sum_{(x,y) \in C \cap M}p(x) + p(y) = \sum_{x \in V(C)} p(x) \].
On the other hand \[
c(C / M) = \sum_{(x,y) \in C / M}c((x,y)) \geq \sum_{x \in V(C)}p(x) \]
This means the cycle is non-negative, which is absurd.

So M is a minimal cost matching.

\SQ{4}

\( P \) starts and ends with an unmatched edge, since \( u \) and \( v \) are not covered by \( M \), so there are \( k \) matched edges and \( k + 1 \) unmatched edges in P. \( M \Delta P \) removes the \( k \) edges from \( M \) and adds the \( k + 1 \) unmatched ones, so it does increase the size of \( M \) by 1. For every edge \( e \in P / M \), the endpoints are either
\begin{enumerate}
\item The endpoints of P, so unmatched
\item The endpoints of the adjacent edges in \( P \) , which are matched since \( P \) is alternating. This means that both these vertices are endpoints of no other edge in \( M \) (they can only be matched once), and the matched edges they were endpoints of is removed from \( M \), so \( e \) is a valid addition to the matching.
\end{enumerate}
As far as the price function \( p \), it has not changed, so clause \( (P1) \) is still verified, and \( \forall e = (x,y) \in M' / M, e \in P\), so \( c(e) = p(x) + p(y) \), as \( P \) is a perfect path. So clause \( (P2) \) is still verified. \(p\) is a price function for \( M \).



\SQ{5}
As long as the current matching is not perfect, there exists an uncovered vertex \( u \).
\begin{enumerate}
\item If \( u \in L \) we have found \( r \).
\item If \( u \in R \), since there are only edges from \( R \) to \( L \), every matched vertex in \( L \) is matched to a vertex in \( R \). Hence, denoting \( L_m \) the matched vertices in \( L \), \( |L_m| = |R_m| < \frac{n}{2} \). So there exists an unmatched vertex \( r \in L \).
\end{enumerate}

\SQ{6}
I will prove the following invariant: at the end of while loops with \( i \) odd, there is an alternating path between \( r \) and every element in \( L_i \) (which is refered to as \( L_{i+1} \) during the loop). \\
Initialisation: \\
i = 1. Every element \( u \in L_1 \) is in it by definition because \( (r,u) \in E \) ( and \( (r,u) \in E / M \) or \( r \) would be matched. So there in in fact an alternating path. \\
Heredity: \\
Let \( i = 2k + 1 \),  Let \( u \in L_i \), by definition of \( L_i \), \( \exists v \in L_{i - 1} \) s.t. \( (u,v) \in E \). 
By definition of \( v \in L_{i-1}, \exists w \in L_{i-2} \) s.t. \( (w,v) \in M \). This means \( v \) is matched and \( (u,v) \not \in M \).
We know from the invariant that there is an alternating path \( P \) between \( r \) and every vertex in \( L_{2k - 1} \), particularly \( w \).
Appending these two edges to \( P \) we get \( P' \) alternating path between \( r \) and \( u \), which concludes the proof. \\

We can only discover an unmatched vertex on a step ending with an odd \( i \), since the other ones only discover vertices throught edges in \( M \), so matched vertices. When such a vertex is discovered, there exists an alternating path from \( r \) to it, and they are both unmatched. Since we considered only tight edges, P is a good path.

\SQ{7}
Every layer is defined by adding only vertices that are not in any of the other layers; this means that they are all disjoint.

Let \( A \) be the union of the odd layers.
If \( x \in A \), the layers are disjoint so \( x \not \in S \). \( x \) is in some odd layer \( i \), which means the layer \( i - 1 \) is in the tree, and by definition of the layer \( i \), there is a tight edge between some vertex in \( i - 1 \) and \( x \). SO \( x \in N_{tight}(S) \). Hence \( A \subset N_{tight}(S) \). \\

Reciproqually, let \( x \in N_{tight}(S) \), this means \(x \not \in S \) and \( \exists y \in S \) s.t. \( xy \) is tight. \( y \in S \) so there exists a layer \( L_{i} \) s.t. \( y \in L{i} \) with even \( i \). \\
If \( x \not \in A \), this means that, when defining \( L_{i+1} \), \( x \not \in \bigcup {k \leq i}L_k \) and \( \exists x \in L_i \) s.t. \( xy \in E\) and is tight, so \( x \in L_{i+1}\), which is absurd. \(x \in A\), so \( N_{tight}(S) \subset A \) \\

\( A  = N_{tight}(S) \)

\SQ{8}
\begin{itemize}
	\item It is easy to proove by induction that all the even  layers are in \( L \) and all the odd layers are in \( R \), so \( S \subset L \).
	\item The initial vertex \( r \) is uncovered, and part of the 0 layer, so \( S \) has at least one uncovered vertex.
	\item Let \( x \in N_{tight}(S) = A \) and suppose it is not matched to a vertex in \( S \). We know it is not unmatched overall, since the algorithm returned the tree rather than a good path. So it is matched to some vertex \( y \not \in S \), then \( y \in L \) so \( y \not \in A \). Hence, since \( x \in A\), \( x \in L_{i} \) for some odd \( i \), and all the conditions are met for \( y \in L_{i+1} \), so \( y \in S \), which is absurd.

\end{itemize}
This proves the 3 clauses for being a good set, and concludes the question.


\SQ{9}


Every element in \( N_{tight}(S) \) is matched to an element in \(S \), and by propreties of matchings, we know that these elements are all distinct. Hence \( |N_{tight}(S)| \leq |S \cap M| \). We also know that \( S \) contains at least one unmatched vertex, so \( |N_{tight}(S)| < |S| \).

\SQ{10}
By construction of the search tree, which looks throught all the tight edges in the graph, every tight edge connected to \( S \) is in \( A \) (this is the same as before, if it is not, it is in the following layer, which is absurd). 
Hence, all the edges considered are NOT tight; \( \forall xy \in E, x \in S, y \not \in N_{tight}(S), c(xy) > p(x) - p(y) \). This proves \( \alpha > 0 \).

\SQ{11}

Let \( xy \in E \).
If \(x \in S\) or \(y \in S\), we will consider \( x \in S \) w.l.o.g. In this case, either:
\begin{itemize}
	\item \( y \in N_{tight}(S) \) and \( c(xy) = p(x) + p(y) \). This means \( p'(x) + p'(y) = p(x) + \alpha + p(y) - \alpha = c(xy) \). The edge stays tight, so if satisfies (P1), and if it was matched, (P2).
	\item \( y \not \in N_{tight}(S) \), which means \( \alpha \leq c(xy) - p(x) - p(y) \). Hence \( p'(x) + p'(y) = p(x) + \alpha + p(y) \leq c(xy) \). So the edge satisfies (P1).
\end{itemize}

If \( x,y \not \in S\) then either
\begin{itemize}
	\item xy is not matched, then \( p'(x) + p'(y) \leq p(x) + p(y) \leq c(xy) \) and (P1) is satisfied.
	\item xy is matched, in which case \( x,y \not \in N_{tight}(S) \) either (since they are all matched to S), so \( p'(x) + p'(y) = p(x) + p(y) = c(xy) \) and both (P1) and (P2) are satisfied.

\end{itemize}

\( p' \) is in fact a price function.

\SQ{12}

Every time the algorithm returns a search-tree path, the matching augments by 1. The matching is bounded in size by \( |L| \), so, if this step is called enought, the algorithm will find a maximal matching and terminate. For this step not to be called a finite amout of times, the search for a good path algorithm would need to run infinite times, and only return a search tree (every time it returns a search tree-path, the matching increases). \\

But returning the search tree leads to a change in price function, which we have proven to be valid, and if we look at the quantity \( \sum_{x \in V}p(x) \), it is strictly increasing at each call of \emph{update the price function}

Note the following, with regard to the new price function:
\begin{align*}
	\sum_{x \in V}p'(x) &= \sum_{x \in S}(p(x) + \alpha) + \sum_{x \in N_{tight}(S)}(p(x) - \alpha) + \sum_{x \in V / S / N_{tight}(S)}p(x)   \\ &= \sum_{x \in V}p(x) + \alpha( |S| - |N_{tight}(S)|) 
\end{align*}
		But we showed that \( |S| > |N_{tight}(S)| \), so \( \Delta(p) = \sum_{x \in V}p'(x) - \sum_{x \in V}p(x) \geq \alpha \). Every weight in the graph is an integer, so, by induction, \( \alpha \in \mathbb{N}^* \), which means \( \Delta(p) \geq 1 \)



For a given graph \( G \), matching \( M \) and valid price function \( p \), define \( A = \sum_{x \in V} p(x) + |M| \). Each run of the algorithm increases this quantity by at least 1.

During a run of the main while loop, the algoithm returns either:
\begin{itemize}
	\item A good path \( P \), we have shown that this path can be used to increase the matching by one edge, with \( p \) still begin a valid price function. Hence \( A \) increases by 1.
	\item A search tree. We have shown we can use this tree to define a new price function, which is still valid for the matching, we also showed that \( \Delta(p) \geq 1 \), so \( A \) increases by at least 1.
\end{itemize}

Lastly, \( sum_{xy \in E}c(xy) \geq sum_{xy \in E}p(x) + p(y) \geq sum_{xy \in E}p(x) + sum_{xy \in E}p(y) \geq \sum_{z \in V}p(z) \) because there exists a perfect mathching, so every vertex has at least one edge connected to it. \( |M| \) is bounded by \( |L| \).
This means \( A \) is bounded and increases by at least one on every loop of the algorithm, so the algorithm terminates. Since the end conditon is for \( M \) to be a perfect matching, it is correct.

\SQ{13}

Let \( n = |E| \) and \( m = |V| \).
Since \( \alpha \) is defined so that \( p' \) is still a price function, but as a minimum, there exists \( xy \in E, x \in S, y \not \in N_{tight}(S) \)  s.t. \( c(xy) - p(x) - p(y) = \alpha \). This means, when updating the price function, an extra edge becomes tight. As we showed int he previous questions, tight edges of the graph stay tight. This means the \emph{update the price function} step can only happen \( O(m) \) times, every time, it will need to look throught \( O(m) \) edges to find \( \alpha \). \\
Updating the matching will happen exactly once for every edge of the matching, or \( O(n) \) times. Finding the search tree is, in the worst case, over all the vertices of the graph, so it's complexity is \( O(n) \). Checking wether \( M \) is maximal is simply a cardinality question, so \( O(1) \).

So the overall complexity of the algoritm is \( O((n+m)^2) \). Since the graph has a perfect matching, \( |E| \geq \frac{|V|}{2} \) so the complexity is \( O(m^2) \).


\end{document}
